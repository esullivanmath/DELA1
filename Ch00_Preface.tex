\setcounter{chapter}{-1}
\chapter{To the Student and the Instructor}
This document contains lecture notes, classroom activities, examples, and challenge
problems specifically designed for a first semester of differential equations and linear
algebra taught with a focus on mathematical modeling. The content herein is written and
maintained by Dr. Eric Sullivan of Carroll College.  Problems were either created by Dr.
Sullivan, the Carroll Mathematics Department faculty, part of NSF Project Mathquest, part
of the Active Calculus text, or come from other sources and are either cited directly or
cited in the \LaTeX\,source code for the document (and are hence purposefully invisible to
the student).


\section{An Inquiry Based Approach}
\begin{problem}[Setting The Stage]
    \begin{itemize}
        \item Get in groups of size 3-4.
        \item Group members should introduce themselves.
        \item For each of the questions that follow I will ask you to:
            \begin{enumerate}
                \item {\bf Think} about a possible answer on your own
                \item {\bf Discuss} your answers with the rest of the group
                \item {\bf Share} a summary of each group's discussion
            \end{enumerate}
    \end{itemize}
    {\bf Questions:} 
    \begin{description}
        \item[Question \#1:] What are the goals of a university education?
        \item[Question \#2:] How does a person learn something new?
        \item[Question \#3:] What do you reasonably expect to remember from your courses
            in 20 years?
        \item[Question \#4:] What is the value of making mistakes in the learning process?
        \item[Question \#5:] How do we create a safe environment where risk taking is
            encouraged and productive failure is valued?
    \end{description}
\end{problem}
(The previous problem is inspired by Dana Ernst's first day activity in IBL activity
titled: 
\href{http://danaernst.com/setting-the-stage/}{Setting the Stage}.)

\begin{quote}
    ``Any creative endeavor is built in the ash heap of failure.'' \\ --Michael Starbird
\end{quote}


This material is written with an Inquiry-Based Learning (IBL) flavor. In that sense, this
document could be used as a stand-alone set of materials for the course but these notes
are not a {\it traditional textbook} containing all of the expected theorems, proofs,
examples, and exposition. The students are encouraged to work through problems and
homework, present their findings, and work together when appropriate. You will find that
this document contains collections of problems with only minimal interweaving exposition.
It is expected that you do every one of the problems and then use other more traditional
texts as a backup when you are stuck.  Let me say that again: this is not the only set of
material for the course.  Your brain, your peers, and the books linked in the next section
are your best resources when you are stuck.

To learn more about IBL go to
\href{http://www.inquirybasedlearning.org/about/}{http://www.inquirybasedlearning.org/about/}.
The long and short of it is that the students in the class are the ones that are doing the
work; building models, proving theorems, writing code, working problems, leading discussions, and pushing the pace. The
instructor acts as a guide who only steps in to redirect conversations or to provide
necessary insight. If you are a student using this material you have the following jobs:
\begin{enumerate}
\item Fight!  You will have to fight hard to work through this material.  The fight is
        exactly what we're after since it is ultimately what leads to innovative thinking.
\item Screw Up!  More accurately, don't be afraid to screw up.  You should write code,
    work problems, and prove theorems then be completely unafraid to scrap what you've
    done and redo it from scratch.  Learning this material is most definitely a non-linear
    path.\footnote{Pun intended: our goal, after all, is really to understand that linear
        algebra is the glue that holds mathematics together.}
        Embrace this!
\item Collaborate!  You should collaborate with your peers with the following caveats:
        (a) When you are done collaborating you should go your separate ways.  When you
        write your solution you should have no written (or digital) record of your
        collaboration.  (b) \underline{The internet is not a collaborator}.  Use of the internet to
        help solve these problems robs you of the most important part of this class; the
        chance for original though.
\item Enjoy!  Part of the fun of IBL is that you get to experience what it is like to
        think like a true mathematician / scientist.  It takes hard work but ultimately
        this should be fun!
\end{enumerate}

\section{Online Texts and Other Resources}\label{pref:resources}
If you are looking for online textbooks for linear algebra and differential equations I
can point you to a few.  Some of the following online resources may be a good place to
help you when you're stuck but
they will definitely say things a bit differently. Use these resources wisely.
\begin{itemize}
    \item The book {\it Differential Equations with Linear Algebra, An inquiry based approach
        to learning} is a nice collection of notes covering much of the material that we
        cover in our class.  The order is a bit different but the notes are well done.
        \\\href{https://content.byui.edu/file/664390b8-e9cc-43a4-9f3c-70362f8b9735/1/316-IBL\%20(2013Spring).pdf}{content.byui.edu/file/664390b8-e9cc-43a4-9f3c-70362f8b9735/1/316-IBL\%20(2013Spring).pdf}
    \item The ODE Project by Thomas Juson is a nice online text that covers many (but
        not all) of the topics that we cover in differential equations. \\
        \href{http://faculty.sfasu.edu/judsontw/ode/html/odeproject.html}{faculty.sfasu.edu/judsontw/ode/html/odeproject.html}
    \item Elementary Differential Equations by William Trench.  This book contains
        everything(!) you would ever want to look up for ordinary differential equations.  It is a
        great resource to look up ODE techniques.  \\
        \href{http://ramanujan.math.trinity.edu/wtrench/texts/TRENCH_DIFF_EQNS_I.PDF}{ramanujan.math.trinity.edu/wtrench/texts/TRENCH\_DIFF\_EQNS\_I.PDF}
    \item A First Course in Linear Algebra by Robert Beezer. This book is very thorough
        and covers everything that we do in linear algebra and much more. \\
        \href{http://linear.ups.edu/fcla/index.html}{linear.ups.edu/fcla/index.html}
    \item Linear Algebra Workbook by TJ Hitchman. This is a workbook for Dr. Hitchman's
        class at U. Northern Iowa.  Even though it is only a ``workbook'' it contains some nice explanations and it has
        embedded executable code for some problems. \\
        \href{http://theronhitchman.github.io/linear-algebra/course-materials/workbook/LinAlgWorkbook.html}{theronhitchman.github.io/linear-algebra/course-materials/workbook/LinAlgWorkbook.html}
\end{itemize}

% \section{Creative Commons License}
% The work in this document is licensed under a Creative Commons Non Commercial Share Alike
% license.  For more information go to:
% \href{https://creativecommons.org/licenses/by-nc-sa/4.0/}{https://creativecommons.org/licenses/by-nc-sa/4.0/}

\section{To the Instructor}
If you are an instructor wishing to use these materials then I only ask that you adhere to the
Creative Commons license.  You are welcome to use, distribute, and remix these materials
for your own purposes.  Thanks for considering my materials for your course!

My typical use of these materials are to let the students tackle problems in small groups
during class time and to intervene when more explanation appears to be necessary or if
the students appear to be missing the deeper connections behind problems.  The course that
I have in mind for these materials is a first semester of differential equations and
linear algebra taught from the standpoint of mathematical modeling.  As such, this is not a complete collection of materials for either
differential equations or linear algebra in isolation.  We discuss
matrix operations, Gaussian elimination, the eigenvalue problem, first order linear
homogeneous and non-homogeneous differential equations, and second order homogeneous
differential equations.  In the second course we will expand upon these ideas and include
more advanced topics.

Many of the theorems in the text come without a proof.  If the theorem is followed by the
statement ``prove the previous theorem'' then I expect the students to have the skill to
prove that theorem and to do so with the help of their small group.  However, this course
is not intended to be a proof-based mathematics course so several theorems are stated
without rigorous proof. If you are looking for a proof-based linear algebra or
differential equations course then I believe that these notes will not suffice.  I have,
however, tried to give thought provoking problems throughout so that the students can
engage with the material at a level higher than just the mechanics of differential
equations and linear algebra.  There are also several routine exercises
throughout the notes that will allow students to practice mechanical skills.

There is a toggle switch in the \LaTeX\ code that
allows you to turn on and off the solutions to problems.  The line of code \\
\verb|\def\ShowSoln{0}|\\
is a switch that, when set to 0, turns the solutions off and when set to 1 turns the
solutions on.  Just re-compile (\texttt{pdflatex}) the document to display the solutions.
I typically do not show the solutions to the students while they're learning the material. 
