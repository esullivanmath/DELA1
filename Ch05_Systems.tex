\chapter{Systems of Difference and Differential Equations}

\section{Rumor Spreading System (Finish)}

\section{Linear Systems (Finish)}

\section{The Eigenvalue / Eigenvector Problem (Finish)}

\section{Analysis of Linear Systems (Finish)}


\section{Modeling Explorations with Systems (FINISH)}
\begin{problem}
We are going to build a difference equation model that predicts the populations of cougars
$C$ and deer $D$ (predators and prey) at year $t$
\begin{enumerate}
    \item {\bf The Mathematical Model:} 
        \begin{enumerate}
            \item {\bf Deer Population:} The change in deer population can be modeled as 
                \[ D'(t) = (\text{logistic population growth}) - (\text{deer removed
                by interactions with cougars}) \]
                \begin{itemize}
                    \item a generic logistic growth difference equation takes the form: 
                        \[ y'(t) = k y \left(1-\frac{y}{N}\right) \]
                        where $k$ is the growth rate parameter and
                        $N$ is the carrying capacity of the population.
                    \item Assume that the carrying capacity for the deer is $N = 4,500$,
                        and assume that $k \approx 0.3$
                    \item The rate at which the deer population is removed due to
                        interactions with cougars is proportional to the product of the
                        deer and cougar populations.  Assume that the proporationality
                        constant is $-0.15$.
                    \item Write the differential equation for the deer population. 
                \end{itemize}
            \item {\bf Cougar Population:} The change in cougar population can be modeled
                as
                \[ C'(t) = \left( \text{natural death rate} \right) + \left(
                    \text{gain due to abundance of food source}
                \right) \]
                \begin{itemize}
                    \item Assume that the natural death rate for the cougar population is
                        proportional to the current number of cougars.  Let the
                        proportionality constant be $-0.30$.
                    \item Assume that the gain in population is proportional to the
                        product of the deer and cougar populations with proportionality
                        constant $0.00015$.
                    \item Write the difference equation for the cougar population.
                \end{itemize}
            \item {\bf The Model:} The two difference equations that you've written along
                with the initial conditions $D_0 = 500$ and $C_0 = 2$ are now your
                mathematical model!  Summarize them in one place in your lab.
        \end{enumerate}
    \item {\bf Model Description:} Describe some features of this model by answering the following questions.
        (please restate the questions in your lab writeup)
        \begin{enumerate}
            \item Which animal is the predator and which is the prey? 
            \item What is the significance of the $+$ sign in the equation for cougars? 
            \item What is the significance of the $-$ sign in the equation for deer? 
            \item In the absence of deer, does the cougar population increase or decrease?
                At what rate?
            \item In the absence of cougars, does the deer population increase or
                decrease? At what rate?
            \item How many deer does each cougar consume when there are 1000 deer vs when
                there are 2000 deer?
        \end{enumerate}

    \item {\bf Numerical Solution:}  Construct a numerical solution to this system of
        difference equations.  Recall that a numerical solution for a differential
        equation model will use Euler's method.  This time you'll need to build both
        equations simultaneously.  Your numerical approximation will give you both the
        deer population and the cougar population.
        For a system of differential equations there are three appropriate plots that you
        need to generate.
        \begin{description}
            \item[Deer vs Time:] Plot the number of deer on the $y$ axis and time ($t$) on the
                $x$ axis.
            \item[Cougars vs Time:] Plot the number of cougars on the $y$ axis and time ($t$) on the
                $x$ axis.
            \item[Deer vs Cougars:] Plot the number of deer on the $y$ axis and the number
                of cougars on the $x$ axis (or visa versa).  This is called the phase
                plot.
        \end{description}
        Write a paragraph describing the behavior of the deer and cougar populations over
        the course of 100 years. Identify each of the key features\footnote{{\it Key
        features} would include maximums, minimums, regions of increase or decrease,
    extinction, overpopulation, equilibrium, etc.} of the population dynamics.

    \item {\bf Analysis:} 
        \begin{enumerate}
            \item There are several assumptions in this model, but in particular we have
                assumed that the cougar population will only go up via direct interactions
                with prey (deer).  Change the death rate term in the cougar population to
                a plausible model that depends on the current number of cougars.  Repeat
                the numerical solutions for several possible parameters.
            \item Is either model (the original or the new from part (a)) sensitive to
                initial conditions?  Vary each initial condition slightly and discuss how
                the dynamical system behaves. Support your disucssion with appropriate
                plots or tables.
            \item There are {\it interaction terms} in each equation of the model (the
                terms involving both $C$ and $D$).  How sensitive is the model to the
                interaction rate $0.15$?  Vary this interaction term slightly and discuss
                the behavior of the dynamical system. Support your discussion with
                appropriate plots or tables.
            \item Attack several other assumptions built into this model, make the
                appropriate changes to the model, and determine what the impact is on the
                resulting populations.
        \end{enumerate}



    \item Write an Executive Summary describing what you did in this lab.  This is meant
        to be a reference to you when you need to re-use the ideas (both mathematical and
        computational) from this lab. Your executive summary should be immediately under
        the title on the front page of the lab.  It should be no more than a paragraph or
        two, and it should make for easy reference later.

\end{enumerate}
\end{problem}


